\textbf{ALESSANDRO E NIKOLETA}

\includegraphics[width=3.53686in,height=4.66895in]{./images/media/image1.png}

\textbf{PRONTI VIAAAAA}

\textbf{GIORNO 1}

Benvenuti in Italia! Vogliamo lasciarvi un ricordo con qualche cenno dei
luoghi che vedrete, in modo che vi resti la voglia di tornare da noi a
trovarci.

\textbf{ARRIVO A BOLOGNA}

Arrivo all'aeroporto \textbf{``Guglielmo Marconi''} e trasferimento
verso il centro città.

Benvenuti nella città dei portici, delle torri e dei grandi sapori!

\textbf{Pranzo nella Trattoria ``Il Meloncello''}

\includegraphics[width=2.31879in,height=2.9409in]{./images/media/image2.png}

Questa storica trattoria è perfetta per un primo assaggio della cucina
bolognese ed è situata ai piedi del colle di San Luca.

\textbf{Piatti della tradizione:}

\begin{itemize}
\item
  Tortellini in brodo
\item
  Tagliatelle al ragù
\item
  Lasagne verdi alla bolognese
\item
  Cotoletta alla bolognese
\item
  Tenerina, zuppa inglese, mascarpone
\end{itemize}

\textbf{Curiosità e particolarità:}

\begin{itemize}
\item
  \textbf{Posizione unica:} accanto all'Arco del Meloncello, dove il
  portico di San Luca compie una curva armoniosa, capolavoro del
  Settecento progettato da Carlo Francesco Dotti.
\item
  \textbf{Legame storico:} la trattoria nasce come punto di ristoro per
  pellegrini e viandanti diretti al Santuario di San Luca.
\item
  \textbf{Cucina autentica:} pasta fresca fatta a mano, ragù
  tradizionale, piatti stagionali.
\item
  \textbf{Atmosfera:} calda e familiare, con fotografie e ricordi che
  raccontano decenni di convivialità.
\end{itemize}

\textbf{Mangiare qui significa essere dentro la storia di Bologna.}

\textbf{PER DIGERIRE UNA BELLA CAMMINATA}

\textbf{Santuario di San Luca}

\includegraphics[width=2.27083in,height=3.35417in]{./images/media/image3.jpeg}\includegraphics[width=2.9375in,height=2.60417in]{./images/media/image4.jpeg}

Situato sul \textbf{Colle della Guardia}, è uno dei luoghi più
riconoscibili e cari ai bolognesi, meta di pellegrinaggi e simbolo della
città.

\textbf{Cenni storici:}

\begin{itemize}
\item
  Le origini risalgono al XII secolo, quando un pellegrino portò a
  Bologna una sacra icona della Madonna, ritenuta miracolosa.
\item
  L'attuale edificio in stile barocco fu completato nel XVIII secolo,
  progettato da Carlo Francesco Dotti, con una grande cupola visibile da
  tutta la città.
\item
  Il Santuario è stato per secoli meta di pellegrinaggi e processioni,
  soprattutto in occasione di calamità o momenti di devozione cittadina.
\item
  La chiesa ospita affreschi e decorazioni che raccontano episodi
  religiosi legati alla Madonna e alla storia di Bologna.
\end{itemize}

\textbf{Il portico più lungo del mondo}

Il \textbf{portico di San Luca} ha una lunghezza complessiva di
\textbf{3,796 km}, ossia quasi 4 chilometri, senza interruzioni.

\begin{itemize}
\item
  \textbf{Numero di arcate:} 666
\item
  \textbf{Scalini:} 489
\item
  \textbf{Cappelle:} 15 dedicate a figure ed episodi religiosi
\end{itemize}

\textbf{Curiosità:}

\begin{itemize}
\item
  I portici di Bologna complessivamente raggiungono \textbf{62 km}, con
  circa 40 km solo nel centro storico.
\item
  Nascono inizialmente come abuso edilizio per aumentare la metratura
  delle abitazioni.
\item
  La legge del 1288 stabiliva che ogni nuovo palazzo doveva avere un
  portico.
\item
  Secondo la leggenda, i 666 archi rappresenterebbero il serpente
  (diavolo) che si scontra con la Madonna del Santuario.
\end{itemize}

\textbf{Particolarità:}

\begin{itemize}
\item
  L'icona della Madonna di San Luca è considerata miracolosa e viene
  portata in processione ogni anno.
\item
  Dal 2021 il Santuario è \textbf{Patrimonio dell'Umanità UNESCO}.
\item
  La posizione panoramica offre una vista spettacolare sulla città e sui
  colli bolognesi.
\end{itemize}

\textbf{Sul percorso:}

\begin{itemize}
\item
  Lo \textbf{Stadio Dall'Ara}, casa del Bologna F.C.
\end{itemize}

\textbf{I portici di Bologna}

\includegraphics[width=1.56214in,height=2.44847in]{./images/media/image5.png}

\textbf{Lunghezza e record:}

\begin{itemize}
\item
  Portico più lungo del mondo: \textbf{San Luca, 3,796 km, 666 archi}
\item
  Estensione totale: 62 km
\item
  Altezza variabile: dal più alto (Palazzo Arcivescovile, oltre 10 m) al
  più stretto (95 cm)
\end{itemize}

\textbf{Storia e leggi:}

\begin{itemize}
\item
  Statuto comunale del 1288: portici alti almeno 2,5 m
\item
  Motivo costruzione: espandere spazi abitativi e proteggere dai
  cambiamenti meteorologici
\item
  Contributo cittadino nel corso dei secoli
\end{itemize}

\textbf{Altre curiosità:}

\begin{itemize}
\item
  Palestra a cielo aperto: i bolognesi li usano per camminata e corsa
\item
  Simbolo di accoglienza e socialità
\item
  Patrimonio UNESCO dal 28 luglio 202
\end{itemize}

\textbf{Centro città -- Alcuni luoghi principali}

\textbf{Basilica di San Petronio ⛪}

\includegraphics[width=2.94505in,height=2.28218in]{./images/media/image6.png}
\includegraphics[width=3.2586in,height=1.8082in]{./images/media/image7.png}

\begin{itemize}
\item
  Gotica, incompiuta, tra le più grandi d'Europa non cattedrali
\item
  \textbf{Meridiana di Cassini:} la più lunga del mondo in un edificio
  chiuso, 67,72 m
\end{itemize}

\begin{quote}
\includegraphics[width=3.03067in,height=1.57447in]{./images/media/image8.png}\includegraphics[width=2.55693in,height=1.58886in]{./images/media/image9.png}
\end{quote}

\textbf{Funzione:} Sulla volta della basilica c'è un piccolo foro detto
foro gnomonico.A mezzogiorno solare, un raggio di sole attraversa il
foro e proietta un'immagine del disco solare sul pavimento. Questa
immagine percorre una lunga linea di marmo che rappresenta il meridiano
della città.

Osservando la posizione del raggio si possono ricavare: il solstizio
d'estate e il solstizio d'inverno, gli equinozi, il passaggio delle
costellazioni zodiacali, l'altezza del sole e la durata del giorno.

Era così precisa da permettere a Cassini di correggere il calendario
gregoriano e studiare le variazioni dell'orbita terrestre.

Perché proprio in San Petronio?

La basilica è così alta e orientata così perfettamente che rappresenta
un ``osservatorio naturale'' ideale.\\
Cassini la scelse perché la distanza tra pavimento e volta garantiva
misurazioni eccezionalmente accurate.

Cosa vedere oggi

Lungo il pavimento troverai:

\begin{itemize}
\item
  la linea meridiana in ottone e marmo,
\item
  le tacche zodiacali,
\item
  i segni dei solstizi,
\item
  le iscrizioni astronomiche originali.
\end{itemize}

È ancora funzionante: nei giorni di sole, verso mezzogiorno, puoi vedere
la macchia luminosa muoversi lentamente sulla linea.

\begin{quote}
\textbf{La facciata di San Petronio è un libro di pietra}
\end{quote}

\begin{itemize}
\item
  I bassorilievi del portale principale, scolpiti da Jacopo della
  Quercia, ispirarono artisti del Rinascimento, compreso Michelangelo.
\end{itemize}

\begin{quote}
🎇 Nel Medioevo qui si facevano spettacoli pirotecnici
\end{quote}

\begin{itemize}
\item
  In piazza venivano rappresentati ``fuochi scenici'' con torri, draghi
  e macchine teatrali: un vero teatro medievale all'aperto.
\end{itemize}

\begin{quote}
🕍 San Petronio non è la cattedrale
\end{quote}

\begin{itemize}
\item
  La cattedrale di Bologna è San Pietro, in via Indipendenza.\\
  Petronio, però, è il patrono della città, e la sua basilica è
  diventata il tempio civico per eccellenza.
\end{itemize}

\begin{quote}
💼 Sotto Piazza Maggiore c'è una Bologna nascosta
\end{quote}

\begin{itemize}
\item
  Una fitta rete di ambienti medievali, magazzini e canalizzazioni corre
  sotto la piazza: alcuni sono visitabili durante aperture speciali.
\end{itemize}

\textbf{Canali sotterranei:}

\includegraphics[width=2.18579in,height=1.91188in]{./images/media/image10.png}\includegraphics[width=2.69906in,height=1.89437in]{./images/media/image11.png}

\begin{itemize}
\item
  Bologna un tempo era attraversata da canali navigabili sotterranei
  chiamati Fossatone o Canali di Bologna. Alcune porte e tratti delle
  mura erano costruiti vicino a questi canali, utilizzati per
  trasportare merci, irrigare orti urbani e alimentare mulini. Oggi
  molti canali sono coperti o nascosti, ma il loro tracciato storico
  rimane visibile in alcune strade e cortili, e nei musei della città.
\end{itemize}

\begin{quote}
🔔 Il Campanile ``fantasma''
\end{quote}

\begin{itemize}
\item
  San Petronio non ha un campanile vero e proprio: la struttura
  progettata non fu mai realizzata. Le campane sono ospitate in una
  torre interna più modesta.
\end{itemize}

\textbf{Voltone del Podestà 🔊}

\includegraphics[width=1.63805in,height=2.54973in]{./images/media/image12.png}
\includegraphics[width=3.33788in,height=2.52984in]{./images/media/image13.png}

Sotto il Palazzo del Podestà si trova un passaggio coperto a crociera
che cela un fenomeno sorprendente.

🌟 Come funziona

Se due persone si posizionano nei quattro angoli diagonali e parlano
rivolte verso il muro, si sentono perfettamente, come se fossero
vicinissime, anche parlando a bassa voce.

🏰 Perché accade

L'effetto è dovuto alle volte a crociera e alla forma ``a imbuto'' degli
archi, che riflettono il suono lungo le pareti.\\
È un caso architettonico rarissimo: un vero ``telefono medievale''.

🩺 A cosa serviva

Secondo la tradizione, i frati lo usavano per:

\begin{itemize}
\item
  confessare lebbrosi e malati contagiosi senza avvicinarsi troppo;
\item
  trasmettere messaggi in segreto durante situazioni delicate.
\item
  Questa funzione ``sanitaria'' è rimasta nella memoria popolare con il
  nome di ``Voltone della Morte'', perché qui sorgeva un antico
  lazzaretto''
\end{itemize}

Piazza del Nettuno 🧜‍♂️

\includegraphics[width=2.56978in,height=2.03517in]{./images/media/image14.png}

A pochi passi da Piazza Maggiore.

💡 Curiosità:

\begin{itemize}
\item
  Il Nettuno del Giambologna (1566) è pieno di allusioni simboliche ai
  poteri del Papa e del cardinale Carlo Borromeo.
\item
  Dal famoso ``punto segreto'', il dito del Nettuno sembra\ldots{}
  diventare qualcosa di più: uno scherzo ottico voluto dallo scultore
  per aggirare la censura.
\item
  La fontana fu la prima della città ad avere un sistema idrico pubblico
  moderno.
\end{itemize}

\textbf{Le Due Torri di Bologna}

\includegraphics[width=2.39454in,height=2.38664in]{./images/media/image15.png}
\includegraphics[width=3.32418in,height=2.36979in]{./images/media/image16.png}

Le due torri sono il simbolo della città. Entrambe pendenti, sono
situate all\textquotesingle incrocio tra le vie che portavano alle
cinque porte. I nomi di Asinelli (la maggiore) e Garisenda (la minore)
derivano dalle famiglie a cui tradizionalmente se ne attribuisce la
costruzione, fra il 1109 ed il 1119. Si ritiene che
l\textquotesingle Asinelli inizialmente fosse molto più alta (i muri in
cima sono di uno spessore che permetterebbe
l\textquotesingle innalzamento di altri 20-25 metri) la sommità che
vediamo oggi è dovuta a un rifacimento di epoca Bentivogliesca (1488)
che la ridusse agli attuali 97,2 m (con uno strapiombo di 2,2 m). Il
Comune ne divenne il proprietario nel XIV secolo e la utilizzò come
prigione e fortilizio. Negli stessi anni intorno alla torre fu
realizzata una costruzione in legno, posta a trenta metri da terra e
unita con una passerella aerea (distrutta da un incendio nel 1398) alla
Garisenda. Si dice che la costruzione fosse voluta da Giovanni Visconti,
Duca di Milano, per tenere meglio d\textquotesingle occhio il turbolento
Mercato di Mezzo (oggi via Rizzoli) e poter sedare per tempo eventuali
rivolte. All\textquotesingle epoca i Visconti avevano preso il potere in
Bologna in seguito alla decadenza della Signoria dei Pepoli, e quindi
erano invisi alla popolazione. In epoca più recente sulla Asinelli fu
installato un ripetitore televisivo della RAI. Durante la seconda guerra
mondiale, tra il 1943 e il 1945, la torre fu utilizzata con funzioni di
avvistamento: quattro volontari si appostavano in cima alla torre
durante i bombardamenti al fine di indirizzare i soccorsi verso i luoghi
colpiti dalle bombe alleate. La Garisenda oggi è alta 48 m e ha uno
strapiombo di 3,2 m, ma inizialmente era alta circa 60 m e fu mozzata
nel 1351 a causa del cedimento delle fondamenta in fase di costruzione
(erroneamente si imputa l\textquotesingle inclinazione a un cedimento
del terreno o a un terremoto). Dal 2023 sono iniziati i lavori alle
torri di Bologna, principalmente per la messa in sicurezza della Torre
Garisenda, a causa di un rischio di crollo dovuto a cedimenti e
instabilità che si sono accentuati nel tempo.
L\textquotesingle obiettivo del cantiere, è di consolidare e restaurare
la torre, utilizzando una metodologia ispirata a quella della Torre di
Pisa, per evitare il collasso entro il 2028. La torre degli Asinelli è
nota in quanto torre pendente più alta d\textquotesingle Italia.

\textbf{Piazza delle Sette Chiese (Santo Stefano)}

\includegraphics[width=3.05294in,height=2.11705in]{./images/media/image17.png}
\includegraphics[width=3.06337in,height=2.10901in]{./images/media/image18.png}

\textbf{Piazza delle Sette Chiese}

\begin{itemize}
\item
  Punto di partenza: qui si vede l'intero complesso.
\item
  Curiosità: chiamata ``piccola Gerusalemme'' perché permetteva il
  pellegrinaggio senza partire.
\item
  Da notare: archi nascosti, portici medievali e facciate sobrie.
\end{itemize}

\textbf{Santo Sepolcro}

\begin{itemize}
\item
  Chiesa centrale, cuore del complesso.
\item
  Curiosità: contiene la \textbf{riproduzione della tomba di Cristo}.
\item
  Dettaglio speciale: pavimenti e colonne servivano per
  \textbf{calcolare feste e solstizi}.
\end{itemize}

\begin{quote}
\textbf{Chiesa del Crocifisso}
\end{quote}

\begin{itemize}
\item
  Custodisce il Calvario.
\item
  Curiosità: chiostri nascosti e decorazioni gotiche nascondono simboli
  medievali.
\item
  Da osservare: archi, colonne e finestre ad arco.
\end{itemize}

\begin{quote}
\textbf{Cortile di Pilato}
\end{quote}

\begin{itemize}
\item
  Luogo di meditazione e simulazione della \textbf{via crucis}.
\item
  Curiosità: la colonna scolpita porta fortuna se toccata senza farsi
  vedere.
\item
  Effetto speciale: eco misterioso tra gli archi.
\end{itemize}

\begin{quote}
\textbf{Chiese minori e chiostri}
\end{quote}

\begin{itemize}
\item
  Spazi di preghiera, meditazione e laboratori dei frati.
\item
  Curiosità: orti e passaggi segreti, con simboli esoterici medievali.
\item
  Da osservare: cortili interni e piccoli archi gotici.
\end{itemize}

\begin{quote}
\textbf{Atmosfera generale}
\end{quote}

\begin{itemize}
\item
  Silenzio sospeso, sensazione di \textbf{spiritualità medievale}.
\item
  Leggenda: luci misteriose e apparizioni durante le processioni
  notturne.
\item
  Punto forte: ogni arco e portico racconta \textbf{fede, storia e vita
  quotidiana dei frati}.
\end{itemize}

Per la piccola sorellina della casa, potrebbe essere interessante farle
scoprire per quando sarà il momento che a Bologna è presente
L'\textbf{Università di Bologna}

\textbf{Università di Bologna}

\includegraphics[width=2.22312in,height=2.02956in]{./images/media/image19.png}
\includegraphics[width=3.22157in,height=2.03536in]{./images/media/image20.png}

L\textquotesingle Università di Bologna, la più antica del mondo
occidentale (1088), è ricca di curiosità:~ha il primo Teatro Anatomico
in legno (Archiginnasio), una vastissima collezione di stemmi araldici
(oltre 6.000), e ha avuto figure chiave
come~\href{https://www.google.com/search?q=Guglielmo+Marconi&sca_esv=61a9446aba78e5ee&sxsrf=AE3TifMgjZaeSV79WANef-Y6CeccCqmRCw\%3A1765355974544&ei=xjE5ad2AIfjt7_UPn8q4iAM&ved=2ahUKEwiU5Prrz7KRAxXIgf0HHUx5IxsQgK4QegQIARAC&uact=5&oq=universit\%C3\%A0+di+bologna+curiosit\%C3\%A0&gs_lp=Egxnd3Mtd2l6LXNlcnAiIXVuaXZlcnNpdMOgIGRpIGJvbG9nbmEgY3VyaW9zaXTDoDIEEAAYHjIGEAAYCBgeMgUQABjvBTIIEAAYgAQYogQyCBAAGIAEGKIEMggQABiABBiiBEjHOFC7HFjqM3ACeAGQAQCYAVugAfQGqgECMTG4AQPIAQD4AQGYAg2gApkHwgIKEAAYsAMY1gQYR8ICBhAAGAcYHsICCBAAGAcYCBgemAMAiAYBkAYIkgcCMTOgB_0zsgcCMTG4B5IHwgcEMC4xM8gHFoAIAA&sclient=gws-wiz-serp&mstk=AUtExfC8xTLJO-tly6PBhBXT_4oDNSBOAXvc8XPL8rvERjDKhqYkAEIgy78EEupIrdQLiPdNAdbFCQnBcwtD0k_y8SF2vaStP_av1Yi9nOo2w7K4k66GPp3fn0Ct1aQi5lfmlBxouH2REZ7CR-HbbmYcJHftU66n2QGNXg6BG29D2nNOufhq3GK3HiNQhqrnzxpd8wf26-UZdq2aQGl5UIy_HyIB6wrK8Q_H3hQPY_oIZCrhVgoE4IPiyEfl-kzJZFu5eqDwkaJWgD00EDaq7OmZhe0f9tZaE8BYWJahJ1cZiXqaCDF2eC9CMEoVjIC0EJT8VldfpS2Gtjri1INMKDUrDnzrIUPAfwTJvRmne0snIKY5&csui=3}{\ul{Guglielmo
Marconi}}~e la prima donna
laureata,~\href{https://www.google.com/search?q=Bettisia+Gozzadini&sca_esv=61a9446aba78e5ee&sxsrf=AE3TifMgjZaeSV79WANef-Y6CeccCqmRCw\%3A1765355974544&ei=xjE5ad2AIfjt7_UPn8q4iAM&ved=2ahUKEwiU5Prrz7KRAxXIgf0HHUx5IxsQgK4QegQIARAD&uact=5&oq=universit\%C3\%A0+di+bologna+curiosit\%C3\%A0&gs_lp=Egxnd3Mtd2l6LXNlcnAiIXVuaXZlcnNpdMOgIGRpIGJvbG9nbmEgY3VyaW9zaXTDoDIEEAAYHjIGEAAYCBgeMgUQABjvBTIIEAAYgAQYogQyCBAAGIAEGKIEMggQABiABBiiBEjHOFC7HFjqM3ACeAGQAQCYAVugAfQGqgECMTG4AQPIAQD4AQGYAg2gApkHwgIKEAAYsAMY1gQYR8ICBhAAGAcYHsICCBAAGAcYCBgemAMAiAYBkAYIkgcCMTOgB_0zsgcCMTG4B5IHwgcEMC4xM8gHFoAIAA&sclient=gws-wiz-serp&mstk=AUtExfC8xTLJO-tly6PBhBXT_4oDNSBOAXvc8XPL8rvERjDKhqYkAEIgy78EEupIrdQLiPdNAdbFCQnBcwtD0k_y8SF2vaStP_av1Yi9nOo2w7K4k66GPp3fn0Ct1aQi5lfmlBxouH2REZ7CR-HbbmYcJHftU66n2QGNXg6BG29D2nNOufhq3GK3HiNQhqrnzxpd8wf26-UZdq2aQGl5UIy_HyIB6wrK8Q_H3hQPY_oIZCrhVgoE4IPiyEfl-kzJZFu5eqDwkaJWgD00EDaq7OmZhe0f9tZaE8BYWJahJ1cZiXqaCDF2eC9CMEoVjIC0EJT8VldfpS2Gtjri1INMKDUrDnzrIUPAfwTJvRmne0snIKY5&csui=3}{\ul{Bettisia
Gozzadini}}, oltre ad essere sempre ai vertici delle classifiche
mondiali.~

\textbf{Curiosità Storiche e Architettoniche}

\begin{itemize}
\item
  \textbf{"Alma Mater Studiorum"}: Fondata nel 1088, è la più antica
  università del mondo occidentale, nata da
  un\textquotesingle iniziativa laica degli studenti.
\item
  \href{https://www.google.com/search?q=Archiginnasio&sca_esv=61a9446aba78e5ee&sxsrf=AE3TifMgjZaeSV79WANef-Y6CeccCqmRCw\%3A1765355974544&ei=xjE5ad2AIfjt7_UPn8q4iAM&ved=2ahUKEwiU5Prrz7KRAxXIgf0HHUx5IxsQgK4QegQIAxAC&uact=5&oq=universit\%C3\%A0+di+bologna+curiosit\%C3\%A0&gs_lp=Egxnd3Mtd2l6LXNlcnAiIXVuaXZlcnNpdMOgIGRpIGJvbG9nbmEgY3VyaW9zaXTDoDIEEAAYHjIGEAAYCBgeMgUQABjvBTIIEAAYgAQYogQyCBAAGIAEGKIEMggQABiABBiiBEjHOFC7HFjqM3ACeAGQAQCYAVugAfQGqgECMTG4AQPIAQD4AQGYAg2gApkHwgIKEAAYsAMY1gQYR8ICBhAAGAcYHsICCBAAGAcYCBgemAMAiAYBkAYIkgcCMTOgB_0zsgcCMTG4B5IHwgcEMC4xM8gHFoAIAA&sclient=gws-wiz-serp&mstk=AUtExfC8xTLJO-tly6PBhBXT_4oDNSBOAXvc8XPL8rvERjDKhqYkAEIgy78EEupIrdQLiPdNAdbFCQnBcwtD0k_y8SF2vaStP_av1Yi9nOo2w7K4k66GPp3fn0Ct1aQi5lfmlBxouH2REZ7CR-HbbmYcJHftU66n2QGNXg6BG29D2nNOufhq3GK3HiNQhqrnzxpd8wf26-UZdq2aQGl5UIy_HyIB6wrK8Q_H3hQPY_oIZCrhVgoE4IPiyEfl-kzJZFu5eqDwkaJWgD00EDaq7OmZhe0f9tZaE8BYWJahJ1cZiXqaCDF2eC9CMEoVjIC0EJT8VldfpS2Gtjri1INMKDUrDnzrIUPAfwTJvRmne0snIKY5&csui=3}{\textbf{\ul{Archiginnasio}}}:
  Sede storica, ospita oltre 6.000 stemmi araldici di studenti (italiani
  e stranieri) e il Teatro Anatomico del 1637, una delle più antiche
  aule di anatomia al mondo.
\item
  \textbf{Il Collegio di Spagna}: Uno dei più antichi collegi
  universitari ancora operativi, fondato nel 1364.
\item
  \textbf{La prima donna laureata}: Si ritiene che Bettisia Gozzadini
  (1209-1261) sia stata la prima donna a laurearsi e a insegnare diritto
  all\textquotesingle università di Bologna,
  cita~\href{https://it.wikipedia.org/wiki/Universit\%C3\%A0_di_Bologna}{\ul{Wikipedia}}.~
\end{itemize}

\textbf{Personaggi Illustri}

\begin{itemize}
\item
  \textbf{Guglielmo Marconi}: Partecipò alle lezioni di Augusto Righi e
  ottenne una laurea ad honorem, legato all\textquotesingle Istituto di
  Fisica dell\textquotesingle ateneo.
\item
  \textbf{Grandi Menti}: Ha attratto e formato geni
  come~\href{https://www.google.com/search?q=Niccol\%C3\%B2+Copernico&sca_esv=61a9446aba78e5ee&sxsrf=AE3TifMgjZaeSV79WANef-Y6CeccCqmRCw\%3A1765355974544&ei=xjE5ad2AIfjt7_UPn8q4iAM&ved=2ahUKEwiU5Prrz7KRAxXIgf0HHUx5IxsQgK4QegQIBhAC&uact=5&oq=universit\%C3\%A0+di+bologna+curiosit\%C3\%A0&gs_lp=Egxnd3Mtd2l6LXNlcnAiIXVuaXZlcnNpdMOgIGRpIGJvbG9nbmEgY3VyaW9zaXTDoDIEEAAYHjIGEAAYCBgeMgUQABjvBTIIEAAYgAQYogQyCBAAGIAEGKIEMggQABiABBiiBEjHOFC7HFjqM3ACeAGQAQCYAVugAfQGqgECMTG4AQPIAQD4AQGYAg2gApkHwgIKEAAYsAMY1gQYR8ICBhAAGAcYHsICCBAAGAcYCBgemAMAiAYBkAYIkgcCMTOgB_0zsgcCMTG4B5IHwgcEMC4xM8gHFoAIAA&sclient=gws-wiz-serp&mstk=AUtExfC8xTLJO-tly6PBhBXT_4oDNSBOAXvc8XPL8rvERjDKhqYkAEIgy78EEupIrdQLiPdNAdbFCQnBcwtD0k_y8SF2vaStP_av1Yi9nOo2w7K4k66GPp3fn0Ct1aQi5lfmlBxouH2REZ7CR-HbbmYcJHftU66n2QGNXg6BG29D2nNOufhq3GK3HiNQhqrnzxpd8wf26-UZdq2aQGl5UIy_HyIB6wrK8Q_H3hQPY_oIZCrhVgoE4IPiyEfl-kzJZFu5eqDwkaJWgD00EDaq7OmZhe0f9tZaE8BYWJahJ1cZiXqaCDF2eC9CMEoVjIC0EJT8VldfpS2Gtjri1INMKDUrDnzrIUPAfwTJvRmne0snIKY5&csui=3}{\ul{Niccolò
  Copernico}},~\href{https://www.google.com/search?q=Erasmo+da+Rotterdam&sca_esv=61a9446aba78e5ee&sxsrf=AE3TifMgjZaeSV79WANef-Y6CeccCqmRCw\%3A1765355974544&ei=xjE5ad2AIfjt7_UPn8q4iAM&ved=2ahUKEwiU5Prrz7KRAxXIgf0HHUx5IxsQgK4QegQIBhAD&uact=5&oq=universit\%C3\%A0+di+bologna+curiosit\%C3\%A0&gs_lp=Egxnd3Mtd2l6LXNlcnAiIXVuaXZlcnNpdMOgIGRpIGJvbG9nbmEgY3VyaW9zaXTDoDIEEAAYHjIGEAAYCBgeMgUQABjvBTIIEAAYgAQYogQyCBAAGIAEGKIEMggQABiABBiiBEjHOFC7HFjqM3ACeAGQAQCYAVugAfQGqgECMTG4AQPIAQD4AQGYAg2gApkHwgIKEAAYsAMY1gQYR8ICBhAAGAcYHsICCBAAGAcYCBgemAMAiAYBkAYIkgcCMTOgB_0zsgcCMTG4B5IHwgcEMC4xM8gHFoAIAA&sclient=gws-wiz-serp&mstk=AUtExfC8xTLJO-tly6PBhBXT_4oDNSBOAXvc8XPL8rvERjDKhqYkAEIgy78EEupIrdQLiPdNAdbFCQnBcwtD0k_y8SF2vaStP_av1Yi9nOo2w7K4k66GPp3fn0Ct1aQi5lfmlBxouH2REZ7CR-HbbmYcJHftU66n2QGNXg6BG29D2nNOufhq3GK3HiNQhqrnzxpd8wf26-UZdq2aQGl5UIy_HyIB6wrK8Q_H3hQPY_oIZCrhVgoE4IPiyEfl-kzJZFu5eqDwkaJWgD00EDaq7OmZhe0f9tZaE8BYWJahJ1cZiXqaCDF2eC9CMEoVjIC0EJT8VldfpS2Gtjri1INMKDUrDnzrIUPAfwTJvRmne0snIKY5&csui=3}{\ul{Erasmo
  da
  Rotterdam}},~\href{https://www.google.com/search?q=Carlo+Goldoni&sca_esv=61a9446aba78e5ee&sxsrf=AE3TifMgjZaeSV79WANef-Y6CeccCqmRCw\%3A1765355974544&ei=xjE5ad2AIfjt7_UPn8q4iAM&ved=2ahUKEwiU5Prrz7KRAxXIgf0HHUx5IxsQgK4QegQIBhAE&uact=5&oq=universit\%C3\%A0+di+bologna+curiosit\%C3\%A0&gs_lp=Egxnd3Mtd2l6LXNlcnAiIXVuaXZlcnNpdMOgIGRpIGJvbG9nbmEgY3VyaW9zaXTDoDIEEAAYHjIGEAAYCBgeMgUQABjvBTIIEAAYgAQYogQyCBAAGIAEGKIEMggQABiABBiiBEjHOFC7HFjqM3ACeAGQAQCYAVugAfQGqgECMTG4AQPIAQD4AQGYAg2gApkHwgIKEAAYsAMY1gQYR8ICBhAAGAcYHsICCBAAGAcYCBgemAMAiAYBkAYIkgcCMTOgB_0zsgcCMTG4B5IHwgcEMC4xM8gHFoAIAA&sclient=gws-wiz-serp&mstk=AUtExfC8xTLJO-tly6PBhBXT_4oDNSBOAXvc8XPL8rvERjDKhqYkAEIgy78EEupIrdQLiPdNAdbFCQnBcwtD0k_y8SF2vaStP_av1Yi9nOo2w7K4k66GPp3fn0Ct1aQi5lfmlBxouH2REZ7CR-HbbmYcJHftU66n2QGNXg6BG29D2nNOufhq3GK3HiNQhqrnzxpd8wf26-UZdq2aQGl5UIy_HyIB6wrK8Q_H3hQPY_oIZCrhVgoE4IPiyEfl-kzJZFu5eqDwkaJWgD00EDaq7OmZhe0f9tZaE8BYWJahJ1cZiXqaCDF2eC9CMEoVjIC0EJT8VldfpS2Gtjri1INMKDUrDnzrIUPAfwTJvRmne0snIKY5&csui=3}{\ul{Carlo
  Goldoni}},~\href{https://www.google.com/search?q=Laura+Bassi&sca_esv=61a9446aba78e5ee&sxsrf=AE3TifMgjZaeSV79WANef-Y6CeccCqmRCw\%3A1765355974544&ei=xjE5ad2AIfjt7_UPn8q4iAM&ved=2ahUKEwiU5Prrz7KRAxXIgf0HHUx5IxsQgK4QegQIBhAF&uact=5&oq=universit\%C3\%A0+di+bologna+curiosit\%C3\%A0&gs_lp=Egxnd3Mtd2l6LXNlcnAiIXVuaXZlcnNpdMOgIGRpIGJvbG9nbmEgY3VyaW9zaXTDoDIEEAAYHjIGEAAYCBgeMgUQABjvBTIIEAAYgAQYogQyCBAAGIAEGKIEMggQABiABBiiBEjHOFC7HFjqM3ACeAGQAQCYAVugAfQGqgECMTG4AQPIAQD4AQGYAg2gApkHwgIKEAAYsAMY1gQYR8ICBhAAGAcYHsICCBAAGAcYCBgemAMAiAYBkAYIkgcCMTOgB_0zsgcCMTG4B5IHwgcEMC4xM8gHFoAIAA&sclient=gws-wiz-serp&mstk=AUtExfC8xTLJO-tly6PBhBXT_4oDNSBOAXvc8XPL8rvERjDKhqYkAEIgy78EEupIrdQLiPdNAdbFCQnBcwtD0k_y8SF2vaStP_av1Yi9nOo2w7K4k66GPp3fn0Ct1aQi5lfmlBxouH2REZ7CR-HbbmYcJHftU66n2QGNXg6BG29D2nNOufhq3GK3HiNQhqrnzxpd8wf26-UZdq2aQGl5UIy_HyIB6wrK8Q_H3hQPY_oIZCrhVgoE4IPiyEfl-kzJZFu5eqDwkaJWgD00EDaq7OmZhe0f9tZaE8BYWJahJ1cZiXqaCDF2eC9CMEoVjIC0EJT8VldfpS2Gtjri1INMKDUrDnzrIUPAfwTJvRmne0snIKY5&csui=3}{\ul{Laura
  Bassi}}~e~\href{https://www.google.com/search?q=Giosu\%C3\%A8+Carducci&sca_esv=61a9446aba78e5ee&sxsrf=AE3TifMgjZaeSV79WANef-Y6CeccCqmRCw\%3A1765355974544&ei=xjE5ad2AIfjt7_UPn8q4iAM&ved=2ahUKEwiU5Prrz7KRAxXIgf0HHUx5IxsQgK4QegQIBhAG&uact=5&oq=universit\%C3\%A0+di+bologna+curiosit\%C3\%A0&gs_lp=Egxnd3Mtd2l6LXNlcnAiIXVuaXZlcnNpdMOgIGRpIGJvbG9nbmEgY3VyaW9zaXTDoDIEEAAYHjIGEAAYCBgeMgUQABjvBTIIEAAYgAQYogQyCBAAGIAEGKIEMggQABiABBiiBEjHOFC7HFjqM3ACeAGQAQCYAVugAfQGqgECMTG4AQPIAQD4AQGYAg2gApkHwgIKEAAYsAMY1gQYR8ICBhAAGAcYHsICCBAAGAcYCBgemAMAiAYBkAYIkgcCMTOgB_0zsgcCMTG4B5IHwgcEMC4xM8gHFoAIAA&sclient=gws-wiz-serp&mstk=AUtExfC8xTLJO-tly6PBhBXT_4oDNSBOAXvc8XPL8rvERjDKhqYkAEIgy78EEupIrdQLiPdNAdbFCQnBcwtD0k_y8SF2vaStP_av1Yi9nOo2w7K4k66GPp3fn0Ct1aQi5lfmlBxouH2REZ7CR-HbbmYcJHftU66n2QGNXg6BG29D2nNOufhq3GK3HiNQhqrnzxpd8wf26-UZdq2aQGl5UIy_HyIB6wrK8Q_H3hQPY_oIZCrhVgoE4IPiyEfl-kzJZFu5eqDwkaJWgD00EDaq7OmZhe0f9tZaE8BYWJahJ1cZiXqaCDF2eC9CMEoVjIC0EJT8VldfpS2Gtjri1INMKDUrDnzrIUPAfwTJvRmne0snIKY5&csui=3}{\ul{Giosuè
  Carducci}}.~
\end{itemize}

\textbf{Curiosità Moderne}

\begin{itemize}
\item
  \textbf{Classifiche Globali}: L\textquotesingle Alma Mater è
  costantemente tra le prime 100 università al mondo e prima in Italia,
  eccellendo in tutte e cinque le macroaree disciplinari (Umanistiche,
  Sociali, Naturali, Mediche, Tecnologiche).
\item
  \textbf{Superstizioni Studentesche}: Come ogni ateneo, ha le sue
  leggende; ad esempio, si dice che non si debbano attraversare certe
  porte per non influenzare il percorso di studi
\end{itemize}

\textbf{Le porte di Bologna}

\includegraphics[width=3.94792in,height=2.12444in]{./images/media/image21.png}

\includegraphics[width=2.81436in,height=1.5in]{./images/media/image22.jpeg}
\includegraphics[width=2.05765in,height=1.67708in]{./images/media/image23.jpeg}

\includegraphics[width=2.48958in,height=1.65708in]{./images/media/image24.jpeg}

Bologna, fin dal Medioevo, era una città fortificata con circa 12 km di
mura e numerose porte, ciascuna con una funzione specifica. Alcune erano
dedicate ai mercanti, altre ai nobili, ai pellegrini o al passaggio del
bestiame. Molte erano dotate di torri di guardia e passaggi sopraelevati
per controllare gli accessi.

Porte principali:

\begin{itemize}
\item
  Porta Saragozza: conduceva al Colle di San Luca, percorso dei
  pellegrini.
\item
  Porta Maggiore e Porta Castiglione: principali accessi commerciali.
\end{itemize}

Curiosità:

\begin{itemize}
\item
  Alcune porte conservano ancora stemmi nobiliari, iscrizioni storiche o
  decorazioni religiose.
\item
  Passeggiando sotto gli archi si può immaginare la vita medievale:
  mercanti, cavalieri e pellegrini che attraversavano la città.
\item
  La costruzione delle porte era spesso collegata ai canali sotterranei
  (Fossatone o Canali di Bologna), utilizzati per trasportare merci,
  irrigare orti urbani e alimentare mulini.
\end{itemize}

Atmosfera attuale:

\begin{itemize}
\item
  Molte porte sono state restaurate e conservano la loro struttura
  medievale.
\item
  Passeggiando tra le porte e le mura si percepisce ancora l'importanza
  strategica e commerciale di Bologna nell'epoca medievale.
\end{itemize}

\textbf{Queste sono solo piccole informazioni per lasciarvi la voglia di
tornare.}

\textbf{E adesso vi lasciamo andare a riposare a Casa Cabe nella
speranza che possiate riposare bene ed essere pronti per il giorno 2 -
13 dicembre: il Matrimonio}

\textbf{GIORNO 2}

Cari amici e familiari, siamo \textbf{orgogliosi e felici} di questa
unione per i nostri ragazzi e auguriamo ad Alessandro e Nikoleta
\textbf{un lungo e felice cammino insieme}, ricco di gioia, amore e
momenti preziosi.\\
Nel cuore della città di Imola, la storica \textbf{Sala Rossa} del
\textbf{Palazzo Comunale} accoglierà il matrimonio di Alessandro e
Nikoleta questa giornata speciale non è solo una celebrazione: è
\textbf{l'inizio della loro unione}, circondati dalla bellezza, dalla
storia e dall'amore che unisce due culture e due cuori.

\textbf{💒 Comune di Imola -- Sala Rossa dove si celebra il matrimonio}

\includegraphics[width=2.94339in,height=2.45833in]{./images/media/image25.png}
\includegraphics[width=2.50593in,height=2.41626in]{./images/media/image26.png}

\textbf{💒 Percorso Nikoleta -- Trasa Nikoletu 💒 Percorso Alessandro --
Trasa Alessandra}

Vogliamo festeggiare i nostri ragazzi con Voi al Ristorante Il Gallo CHE
non solo è un locale storico: MA sarà il luogo dove festeggeremo il
matrimonio di Alessandro e Nikoleta, un momento unico da condividere con
familiari.

\includegraphics[width=2.82699in,height=2.60614in]{./images/media/image27.png}

Per arrivare nella località di Castel del Rio, vi raccontiamo qualche
curiosità sui luoghi che incontreremo lungo il tragitto

\begin{itemize}
\item
  \textbf{🎉 Casalfiumanese --- Sagra del Raviolo}
\end{itemize}

\begin{quote}
\includegraphics[width=2.12781in,height=1.45909in]{./images/media/image28.png}
\includegraphics[width=2.08283in,height=1.43914in]{./images/media/image29.png}

Casalfiumanese si trova nella valle del Santerno ed è un paese noto per
la sua tradizione gastronomica. Ogni anno, in occasione della Sagra del
Raviolo --- tradizionalmente celebrata la domenica più vicina al 19
marzo il paese si anima con una festa che unisce cibo, musica,
spettacoli e grande partecipazione popolare.\\
Il piatto simbolo è un raviolo dolce, ripieno di marmellata o mostarda,
bagnato con liquore e spolverizzato di zucchero: una specialità
dolciaria locale.\\
Il momento più caratteristico è il ``lancio dei ravioli'': i dolci
vengono ``gettati'' sulla folla dalla torre civica verso la piazza
affollata.\\
l'evento non è solo gastronomia: ci sono carri allegorici, spettacoli,
musica un'occasione di festa e comunità per grandi e piccoli.\\
La sagra si tiene ogni anno intorno al 19 marzo, in concomitanza con la
festa di San Giuseppe.\\
Casalfiumanese si trova immersa in un territorio di colline dolci e
campagne, tipico della valle del Santerno.
\end{quote}

\begin{itemize}
\item
  \textbf{🏞️ Borgo} \textbf{Tossignano \& Vena del Gesso Romagnola}
\end{itemize}

\begin{quote}
\includegraphics[width=2.87123in,height=1.56887in]{./images/media/image30.png}
\includegraphics[width=2.06372in,height=1.57471in]{./images/media/image31.png}

Borgo Tossignano sorge nella valle del fiume Santerno, dominata dalla
Vena del Gesso Romagnola, una dorsale di roccia gessosa che attraversa
le colline.\\
La Vena del Gesso è oggi area protetta e offre vari percorsi e sentieri
escursionistici --- ideali per chi ama trekking, natura e panorami sulle
forre e le colline.\\
Il borgo è formato da due nuclei: ``Borgo'', in fondovalle, e
``Tossignano'' sul promontorio gessoso, con una storia di convivenza tra
ambiente fluviale e paesaggio collinare.\\
A Tossignano c'è un centro visite del parco che offre informazioni
sull'ambiente, fauna e flora locali. Ogni anno a Tossignano si celebra
la Sagra della Polenta e la Festa di San Bartolomeo, con cibo, giochi,
musica e mercatini.
\end{quote}

\begin{itemize}
\item
  \textbf{🌿 Fontanelice --- sagre, eventi e curiosità}
\end{itemize}

\begin{quote}
\includegraphics[width=2.25113in,height=1.55566in]{./images/media/image32.png}

Fontanelice si trova nella valle del Santerno, con un tratto fluviale
frequentato dagli abitanti, ideale per relax, picnic e momenti all'aria
aperta.\\
Il borgo è legato all'architetto Giuseppe Mengoni, nato a Fontanelice,
noto per la Galleria Vittorio Emanuele II a Milano. Eventi principali:
Antica Fiera Agricola di fine agosto, Sagra della Piè Fritta a
Pasquetta, Calici di Stelle il 10 agosto e Fiume di Vino lungo il fiume
Santerno. La combinazione di natura, sagre e storia rende Fontanelice
una tappa imperdibile per chi visita la valle del Santerno. Altra nota
degna di essere detta, nel Fontanelice calcio il nostro Matteo
Cornacchia è l'attaccante della squadra da quest'anno.
\end{quote}

\textbf{🏰 CASTEL DEL RIO -- PONTE DEGLI ALIDOSI E FIUME SANTERNO}

\includegraphics[width=2.21291in,height=2.19093in]{./images/media/image33.png}

\textbf{Castel del Rio} è un borgo medievale situato nell'Appennino
bolognese, famoso per il suo \textbf{castello e il ponte monumentale}
che attraversa il fiume Santerno.

\textbf{Ponte degli Alidosi}

\begin{itemize}
\item
  Il \textbf{Ponte degli Alidosi} è un capolavoro dell'ingegneria
  rinascimentale, costruito nel \textbf{1499} per volere della famiglia
  Alidosi, signori del borgo.
\item
  È un ponte \textbf{ad arco ellittico} lungo circa 42 metri, con una
  sola campata che supera il fiume Santerno.
\item
  • \textbf{Cenni storici :}\\
  o La sua costruzione fu necessaria per collegare il borgo con le vie
  commerciali e i centri vicini.
\item
  o La famiglia Alidosi volle un ponte imponente e sicuro, che
  resistesse alle piene del fiume.
\item
  o Oggi è considerato uno dei ponti medievali più eleganti e ingegnosi
  d'Italia.
\end{itemize}

• \textbf{Curiosità :}\\
Il ponte è stato progettato per resistere alle inondazioni del Santerno,
che spesso in passato minacciavano le strade e i campi circostanti.\\
Offre un punto panoramico perfetto per ammirare il borgo e la vallata
circostante.

A FINE SETTMBRE Si svolge tradizionalmente, quando il borgo si anima di
profumi, stand e attività legate al re dei boschi: il fungo porcino. La
Sagra del Porcino di Castel del Rio è una delle manifestazioni
gastronomiche più note dell'Appennino imolese (provincia di Bologna).

\includegraphics[width=2.64184in,height=1.65962in]{./images/media/image34.png}
\includegraphics[width=3.28337in,height=1.59686in]{./images/media/image35.png}

🍄 Protagonista assoluto: il porcino\\
La sagra celebra il fungo locale, raccolto nei boschi dell'Alto
Santerno. Gli stand gastronomici propongono piatti tipici come
tagliatelle ai porcini, bruschette, crescentine e carni accompagnate da
funghi freschi.

\textbf{Fiume Santerno}

\begin{itemize}
\item
  Il fiume nasce nell'Appennino tosco-emiliano e attraversa la valle del
  Santerno prima di confluire nel Reno.
\item
  In passato, il Santerno era fondamentale per \textbf{l'irrigazione dei
  campi, mulini e attività commerciali} lungo le sue sponde.
\item
  Passeggiando lungo il fiume e sul ponte, si possono osservare
  \textbf{tratti di natura incontaminata e scorci medievali} del borgo.
\end{itemize}

\textbf{Suggerimento di visita}

\begin{itemize}
\item
  Passeggiare sul \textbf{Ponte degli Alidosi}, ammirando la struttura
  ad arco e le sue proporzioni armoniose.
\item
  Esplorare il borgo e il castello, che conserva \textbf{sale affrescate
  e arredi storici}.
\item
  Fermarsi sulle rive del Santerno per fotografare il paesaggio e
  ascoltare il \textbf{rumore del fiume}, che accompagna da secoli la
  vita del borgo.
\end{itemize}

FINALMENTE CON I PIEDI SOTTO IL TAVOLO

\textbf{🍷 Ristorante Il Gallo -- Castel del Rio}

\textbf{\hfill\break
}\includegraphics[width=2.88617in,height=1.93009in]{./images/media/image36.png}

\textbf{📍 Dove si trova} • Indirizzo: Piazza della Repubblica 28/30,
Castel del Rio (BO)

• Storico locale di famiglia, gestito dai Franceschelli dal 1947, con
oltre 70 anni di tradizione culinaria.\\
\textbf{🍽️ Cucina e specialità} • Cucina tipica emiliano-romagnola e
appenninica: pasta fresca fatta in casa, carni alla brace, funghi
porcini, castagne locali e piatti stagionali.

Alcuni piatti consigliati: tagliolini ai funghi porcini e guanciale,
tortelloni di ricotta al burro e salvia, tagliata di manzo, funghi alla
brace o fritti.

Ogni stagione offre un menu diverso, legato ai prodotti locali e alla
tradizione della zona.

\emph{SE FINITO DI MANGIARE ABBIAMO ANCORA QUALCHE ENERGIA CI PIACEREBBE
PORTARVI IN QUESTO PICCOLO BORGO CHE MERITA ATTENZIONE PER LE SUE
PARTICOLARITA' CHE VI ACCENNIAMO}

\textbf{🏰 DOZZA -- IL BORGO DIPINTO}

\includegraphics[width=4.10474in,height=2.27115in]{./images/media/image37.png}

\textbf{Dozza} è un affascinante borgo medievale situato a circa 30 km
da Bologna, noto per i suoi \textbf{murales colorati} che decorano le
facciate delle case.

\textbf{Cenni storici}

\begin{itemize}
\item
  Le origini di Dozza risalgono al Medioevo; il borgo si sviluppò
  attorno al \textbf{Castello Sforzesco}, oggi visitabile.
\item
  Nel 1960 nasce la \textbf{Biennale del Muro Dipinto}, iniziativa che
  invita artisti italiani e internazionali a dipingere le facciate delle
  case del borgo.
\item
  Il borgo conserva ancora \textbf{strade lastricate, torri e palazzi
  antichi}, che si intrecciano con i murales moderni, creando un mix
  unico di storia e arte contemporanea.
\end{itemize}

\textbf{I murales}

Ogni anno Dozza accoglie nuovi dipinti, con temi che spaziano da
\textbf{scena quotidiana, fantasia, politica, storia locale}.

\begin{itemize}
\item
  I dipinti raccontano storie del borgo e della regione, spesso con
  \textbf{colori vivaci e dettagli sorprendenti}.
\item
  Passeggiando tra le stradine, si possono osservare \textbf{firme degli
  artisti e date dei murales}, quasi come leggere un libro a cielo
  aperto.
\end{itemize}

\textbf{Curiosità}

\begin{itemize}
\item
  Il borgo ospita anche la \textbf{Cantina di Dozza}, dove si possono
  degustare vini locali come il \textbf{Sangiovese di Romagna}.
\item
  Alcuni murales sono \textbf{interattivi}, invitando i visitatori a
  scattare foto creative o a cercare piccoli dettagli nascosti.
\item
  Il Castello di Dozza ospita una \textbf{collezione di opere d'arte} e
  mostre temporanee legate alla cultura locale.
\end{itemize}

\textbf{Suggerimento di visita}

\begin{itemize}
\item
  Passeggiare senza fretta per le vie principali (Via XX Settembre, Via
  Borgo, Piazza Zotti), osservando \textbf{ogni facciata dipinta}.
\item
  Entrare nel \textbf{Castello Sforzesco} per vedere gli affreschi
  storici e ammirare il borgo dall'alto.
\item
  Fermarsi a degustare \textbf{vino e prodotti tipici} nelle cantine e
  nei negozi locali.
\end{itemize}

\textbf{GIORNO 3}

\textbf{CIRCUITO DI IMOLA -- VISITA A PIEDI}

\includegraphics[width=2.97721in,height=1.65961in]{./images/media/image38.png}
\includegraphics[width=2.52962in,height=1.62942in]{./images/media/image39.png}

\textbf{🏎️ IMOLA}

Il Circuito Internazionale Enzo e Dino Ferrari è famoso in tutto il
mondo per le gare di Formula 1, MotoGP e altre competizioni
motoristiche. Anche senza salire in auto, è possibile fare un tour a
piedi per scoprirne la storia e i punti simbolici.

\textbf{Cenni storici}

Il circuito è stato inaugurato nel 1953, inizialmente come pista per
gare locali e nazionali.

\begin{itemize}
\item
  Nel 1980 è stato intitolato a Enzo Ferrari e a suo figlio Dino, grandi
  protagonisti della storia automobilistica italiana.
\item
  Ha ospitato gare di Formula 1, MotoGP, Superbike e manifestazioni di
  livello internazionale.
\end{itemize}

\textbf{Percorso a piedi}

Partenza: Ingresso principale del circuito, vicino al paddock e alla pit
lane.

Passeggiare lungo la pit lane, osservando le postazioni dei team e le
scuderie. Arrivare al traguardo e alla torre dei box, simbolo di tutte
le competizioni. Proseguire verso la Curva Tosa e la Variante Alta,
punti iconici che hanno visto gare memorabili. Ammirare le tribune e i
paddock storici, dove si respira l'atmosfera dei grandi eventi
motoristici. Percorrere parte del rettilineo principale fino a tornare
all'ingresso, completando il circuito a piedi.

\textbf{Curiosità}

Il circuito di Imola è uno dei pochi in Europa costruiti intorno a una
città, con curve strette e rettilinei veloci. La Variante Acque Minerali
prende il nome dalle sorgenti locali che attraversano la zona.Ogni curva
ha una storia: qui hanno corso campioni come Ayrton Senna, Michael
Schumacher, Valentino Rossi.Durante il tour a piedi si possono vedere le
statue e le targhe commemorative dei piloti che hanno fatto la storia
del circuito.

\textbf{Parco Acque Minerali}

\includegraphics[width=3.25337in,height=1.83509in]{./images/media/image40.png}

Il Parco delle Acque Minerali è il polmone verde storico di Imola.
Prevalentemente conosciuto per la sua funzione ricreativa e sportiva,
presenta anche un indubbio valore botanico e storico.\\
Si trova a poca distanza dal centro della città ed è oggigiorno
interamente circondato dall'Autodromo ``Enzo e Dino Ferrari''. Ebbe
origine dalla scoperta del dott. Gioacchino Cerchiari, avvenuta nel
1830, delle sorgenti curative di acque sulfuree che resero il luogo
immediatamente popolare.

~

Risale invece al 1871 la prima sistemazione dell'area in un vero e
proprio parco, con la realizzazione di viali e aiuole secondo il modello
detto ``all'inglese''. Attualmente il parco ha un'estensione di 11
ettari ed ha un ricco patrimonio di specie arboree, sia autoctone che
esotiche. I recenti interventi hanno inteso valorizzare tale patrimonio
dotando il parco di due aree giochi e di un percorso didattico a
carattere geologico, e riqualificando alcune aree storiche: la zona
delle antiche sorgenti (segnalata dal restauro delle cisterne
originali), la scalinata monumentale che da viale Atleti Azzurri
d'Italia conduce al Belvedere, e gli ingressi al parco.

\ldots{} e se abbiamo ancora energie, forza andiamo nel \textbf{borgo di
Brisighella}

\textbf{Un po' di storia e ambientazione}

Brisighella si trova in Emilia‑Romagna, nella valle del Lamone, ed è
immersa nella natura dell'Appennino tosco-romagnolo. Il borgo è
costruito attorno a tre caratteristiche ``pinnacoli'' di gesso (``i tre
colli''), su cui poggiano tre monumenti simbolo: la fortezza, la torre
dell'orologio e il santuario. Nel tempo Brisighella ha mantenuto un
centro storico ben conservato, con vie medievali acciottolate, antiche
mura, scalini nella roccia e un intreccio di vicoli e passaggi che
raccontano secoli di storia. Il territorio intorno è parte del Parco
Regionale della Vena del Gesso Romagnola: colline, ``gessi'', calanchi,
grotte e sentieri --- un legame tra natura, geologia e storia del borgo.

\textbf{Rocca Manfrediana (Rocca di Brisighella)}

È la fortezza medievale del borgo, costruita nel 1310 per volere della
famiglia Manfredi di Faenza. Restaurata nei secoli, oggi è visitabile:
dentro ci sono due torri, un piccolo museo, e camminamenti che offrono
una vista stupenda su tutto il borgo e la valle. Camminare tra le mura è
come fare un salto nel tempo: si percepiscono l'importanza strategica
del luogo e la sua storia difensiva.

\textbf{Torre dell\textquotesingle Orologio di Brisighella}

La torre sorge su uno dei tre colli, ed è parte dell'antico sistema
difensivo del borgo: la sua costruzione risale al 1290. Nel 1850 venne
ricostruita e fu installato l'orologio: il quadrante originale utilizza
il sistema orario italiano antico (solo sei ore), un dettaglio curioso
che richiama il suo passato. Dalla sommità della torre si gode un
panorama memorabile sul paese, sulle colline di gesso e sui calanchi:
vale la salita!

\textbf{Via degli Asini (antica ``Via del Borgo'')}

Questa via, risalente al Medioevo, è un percorso sopraelevato e coperto
--- un tempo usato per far transitare asini e birocciai che
trasportavano gesso dalle cave. Oggi è una passeggiata suggestiva tra
archi e balconi, che attraversa il cuore del centro storico, rendendo
l'atmosfera del borgo unica e molto caratteristica.

\textbf{Pieve di San Giovanni in Ottavo (Pieve del Thò)}

È una basilica in stile romanico, con tre navate, le cui origini
risalgono forse all'VIII--X secolo, anche se viene documentata dal 909.
All'interno sono emerse tombe antiche (romane o paleo‑cristiane), a
testimonianza di una frequentazione del luogo molto più antica --- segno
che questa zona è abitata da secoli. È un luogo perfetto per chi ama
arte e storia antica, ben diverso dal tono più militar‑medievale di
castelli e torri.

\textbf{santuario della Madonna del Monticino}

Edificato nel XVIII secolo sulla collina di Monticino, sormonta il borgo
e offre una vista magnifica sui dintorni. La chiesa è nata per
proteggere un'icona della Madonna, venerata da secoli nella zona:
rappresenta la dimensione spirituale e devozionale del paese.
Raggiungerla è anche un modo per vivere la natura e la tranquillità
delle colline intorno a Brisighella.

\textbf{Natura, dintorni e esperienze ``fuori porta''}

Brisighella si trova all'interno del Parco della Vena del Gesso
Romagnola, un paesaggio irripetibile fatto di rocce di gesso, calanchi,
colline boschive, grotte e sentieri: perfetto per trekking, passeggiate
e amanti della natura. A pochi chilometri dal centro si trovano luoghi
come la Grotta Tanaccia e il centro visite Ca\textquotesingle{} Carnè,
punti di interesse naturalistico del parco. Nei dintorni, ci sono anche
altri borghi e piccoli paesi da scoprire, oppure percorsi nelle colline
che offrono viste spettacolari e tranquillità immersa nella natura.

\textbf{E ADESSO VIA PER L'ARABIA PER VIVERE INSIEME E COSTRUIRE LA
VOSTRA VITA INSIEME.}

\includegraphics[width=4.16725in,height=2.06279in]{./images/media/image41.png}
